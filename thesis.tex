% Created 2019-05-13 h 18:30
\documentclass{elteikthesis}[2018/06/06]
\usepackage[utf8]{inputenc}
\usepackage[T1]{fontenc}
\usepackage{fixltx2e}
\usepackage{graphicx}
\usepackage{longtable}
\usepackage{float}
\usepackage{wrapfig}
\usepackage{rotating}
\usepackage[normalem]{ulem}
\usepackage{amsmath}
\usepackage{textcomp}
\usepackage{marvosym}
\usepackage{wasysym}
\usepackage{amssymb}
\usepackage{hyperref}
\tolerance=1000
\usepackage{minted}
\date{2019}
\degree{Computer Science BSc}
\supervisor{Zoltan Porkolab}
\affiliation{Associate Professor, PhD}
\university{Eötvös Loránd University}
\faculty{Faculty of Informatics}
\department{Department of Programming Languages}
\departmentSecondLine{ And Compilers }
\city{Budapest}
\logo{elte_cimer_szines}
\documentlang{english}
\newcommand*{\code}{\lstinline[keywordstyle=\color{violet}, basicstyle=\color{violet}]}
\usepackage{endnotes}
\renewcommand{\footnote}{\endnote}
\lstset{numbers=left, stepnumber=1}
\author{Ia Mgvdliashvili}
\date{\today}
\title{Test Case Generation Based On Fuzzing For C++}
\hypersetup{
  pdfkeywords={},
  pdfsubject={},
  pdfcreator={Emacs 25.2.2 (Org mode 8.2.10)}}
\begin{document}

\maketitle
\setcounter{page}{1} \\
\setcounter{tocdepth}{0}
\tableofcontents

\chapter{Introduction}
\label{sec-1}
Most software heavily relies on unit tests as its primary source for logic and fault tolerance verification. This approach has been largely considered as essential, but it has some inherent difficulties associated with it. Although testing single member functions independently is more often than not trivial, most of the time the user will call various combinations of them. It is impossible to write unit tests with all possible function call sequences since such space is effectively infinite. Therefore, the need arises for the developer to personally determine which function call sequences are most meaningful. \\

Other than that, a lot of times the behavior of the function will depend on internal state of the instance, which is in itself reached after certain function calls. \\
\section{Background}
\label{sec-1-1}
\subsection{Fuzzing}
\label{sec-1-1-1}
Dynamic analysis, or fuzzing, is a popular and effective method of finding vulnerabilities in software. Fuzz testing reaches impressive results in exposing interface vulnerabilities in very short amount of time. \\

Fuzzing heavily relies on the concept of Fuzz target - a function that accepts an array of bytes and then uses it in user defined way against the API under test. This API usually has a single endpoint that consumes any kind data. Anything that causes an exception, abort, exit, crash, assert failure, timeout is considered a bug\footnote{DEFINITION NOT FOUND.}. That means, discovery of the first instance of any one of them will cause the fuzzer to halt and inform us about the input that caused the bug, along with some other information. \\

There are a number of tools available for fuzzing, including AFL and Radamsa. One of the most notable implementations is \code{LibFuzzer}, LLVM's tool for coverage guided, evolutionary fuzzing engine\footnote{DEFINITION NOT FOUND.}. The code coverage information for \code{LibFuzzer} is provided by \code{SanitizerCoverage} instrumentation, and I will discuss it in the next subsection. \\
\subsection{SanitizerCoverage library}
\label{sec-1-1-2}
LLVM has an interface for its built-in code coverage instrumentation\footnote{DEFINITION NOT FOUND.}. The user is able to gather information about the covered regions of the program during runtime. There are several different levels of depth for coverage, and the library also offers rich ways to trace the data flow. This tool was crucial for the development of my program and in the developer's manual, I discuss the library in more detail. \\

\section{Using fuzz testing for unit tests}
\label{sec-1-2}
Although fuzz testing has been mostly defined to be for exploiting the vulnerabilities of the program, I tried to apply its coverage based philosophy to explore the possible member function call sequences and pinpoint ones which might be most interesting for the developer. \\
This also required to change the overall approach with which fuzzing is used. \\

In \ref{sec-1-1-1} Background on Fuzzing, I talked about the classical assumptions about the fuzz target. In our scenario, we have different expectations - since we are testing an entire unit and not a single API endpoint, some kind of control flow disruptions might be expected. For example, assertions are common in member functions. Therefore, the previous approach of exiting on first such failure should be modified to allow the program to gather information about all possible combinations that result in things like exceptions, so the user will be informed about them and decide what constitutes the normal behavior of their library and what is outside of specifications. \\

\section{Program description}
\label{sec-1-3}
To achieve the intended results, I created a program that uses LLVM's \code{SanitizerCoverage} library and generation based fuzzing. The test case needs almost minimal setup which consists of the user specifying all the member functions it wants to use in testing, and passing a single function pointer for constructing an instance of the class. Modern C++ tools have aided greatly with this by giving the ability to store pointers to functions with different type signatures. There are still difficulties with regards to determining and passing the function arguments, which is in scope of a larger project. In order for this issue to not interfere with the initial program implementation, I allow users to additionally pass pointers to the functions that will in turn call the specific member function with desired arguments. \\

\section{Results}
\label{sec-1-4}
Although the original intention was to discover new test cases, there were some surprising outcomes that could not have been anticipated. For example, the program is very good in minimizing the total number of test cases, since it only saves the new sequence of function calls if new coverage was discovered, or if it replaces one or more other sequences by combining their result together. This outcome would be crucial for reducing the size of test suites, which leads to reduced runtime and maintenance cost. \\
\chapter{User Documentation}
\label{sec-2}
This software aids in finding test cases for a single class interface by trying out different permutations of member function calls and discovering ones that give the most coverage. In this chapter, I provide full information about its usage for users of the program. I will be using a simple stack implementation as an example\footnote{DEFINITION NOT FOUND.}. \\
\section{Intended audience}
\label{sec-2-1}
This software is intended for C++ developers who would like to discover meaningful function call sequences that provide good coverage for their interface/class. Therefore, at least basic knowledge of C++ is assumed, and the user will need to implement and pass pointers for several functions. \\

\subsection{Requirements}
\label{sec-2-1-1}
Some restrictions apply with regards to the test target class. It should satisfy following for the program to work properly: \\
\subsubsection{The program is intended to test a single unit}
\label{sec-2-1-1-1}
Current version can not analyze any of the dependencies of the class in some cases. Although the feature might be able to work with multiple classes and different member functions, only single one is supported at this stage. \\
\subsubsection{You need to be able to be compiled separately}
\label{sec-2-1-1-2}
In order to analyze , the object file of the test target needs to be compiled with special flags separately from the rest of the project. This means the implementation of the class can not be spread in multiple files, and those files should not contain anything else \\
\section{System Requirements}
\label{sec-2-2}
Following dependencies need to be installed in order to \\
\begin{itemize}
\item GNU Make 4.0+ \\
\item Clang 7.0+ with AddressSanitizer (included with the compiler by default) \\
\end{itemize}
\section{Installation instructions}
\label{sec-2-3}
run these commands from the directory where you have copied the contents of the project and run commands provided in Listing \ref{Installation-commands} \\
\begin{listing}
\caption{\label{Installation-commands}Installation commands}
cd <dir> \# where dir is the path of the folder \\

make test-main \\

make integration-test-run \\
\end{listing}
After tests pass successfully, you can move on to next step and set up the project to test your class. \\
\section{Instructions using an example}
\label{sec-2-4}
After installation, and successful tests, you need to modify the contents of \code{src/main.cc} file and several lines in \code{Makefile}. I will demonstrate the instructions using an example. \\
\begin{listing}
\caption{\label{stack-definition}definition for the stack class in \code{examples/stack.h}}
\inputminted[firstline=6,lastline=24]{c++}{examples/stack.h} \\
\end{listing}
\subsection{Sample stack  class}
\label{sec-2-4-1}
Listing \ref{stack-definition} shows the definition for the class that the project tests by default. Full implementation can be found in corresponding \code{.cpp} file of the same directory. I will go through the example and explain how it can be adjusted for any other class. \\
\subsection{Structure of the main file}
\label{sec-2-4-2}
The user is advised to only change the contents of \lstinline{main()} function, and replace the \lstinline{#include} directive. I will explain how to construct the \code{CombinationTester} class instance and all necessary dependencies. More details about how classes used in the \code{main.cc} file work and way they are engineered can be found in Developer documentation. \\
\begin{listing}
\caption{\label{makefile-user-variables}defining user variables in \code{Makefile}}
\inputminted[firstline=14,lastline=15]{bash}{Makefile} \\
\end{listing}

\begin{listing}
\caption{\label{main-file}Contents of \code{src/main.cc}}
\inputminted[firstline=28,lastline=52,breaklines]{c++}{src/main.cc} \\
\end{listing}

Listing \ref{main-file} shows how the \code{main.cc} file should look like. You shoud replace all template arguments with your class instead of \code{stack<int>}. \\

Follow these steps to set up the \code{main.cc} file: \\
\subsubsection{1. Include your header file and provide it's path}
\label{sec-2-4-2-1}
In the beginning of \code{Makefile}, set the \code{$(TEST_TARGET_FILE)} and \code{$(TEST_TARGET_INC)} variables to point to correct file and path, respectively. For example, the stack class located in \code{examples} directory of the project path would be set as shown on Listing \ref{makefile-user-variables}. \\

\subsubsection{2. Create a function that constructs an instance of your class}
\label{sec-2-4-2-2}
This function will be called in the beginning of testing each function call sequence. \code{int sequenceLength} will be passed and you can use this value if it's applicable. \\
\subsubsection{3. Insert pointers to member functions}
\label{sec-2-4-2-3}
Choose which functions you would like to use for testing. Feel free to omit ones that are not crucial to coverage, for example getters. For functions that require arguments, you need to pass pointers (or simply lambda functions) \\
\subsubsection{4. Construct the CombinationTester class}
\label{sec-2-4-2-4}
You need to pass the objects that you created in steps 2 and 3. Feel free to tweak the maximum number of combinations to suit your library's needs. If the functions are relatively small but there is a greater number of them, it is more reasonable to settle for numbers lower than 5. In some cases large size of functions requires a very specific state of the class which can only be achieved by more function calls. Such cases will better be resolved by mutation-based fuzzing. \\
\subsubsection{5. Choose the output format}
\label{sec-2-4-2-5}
After finishing, you can ask the \code{coverageReporter} to show results by simply printing it or writing to a file \\

\subsection{Compile and run}
\label{sec-2-4-3}
If you followed the instructions in the previous section, you are ready to generate test cases for your class using \code{`make run`} command. \\
\subsubsection{Understanding the output}
\label{sec-2-4-3-1}
After the program is done running, it will display the results as you indicated in the end of \code{main.cc} file. The example is provided in Listing \ref{stack-output}, 3 different sequences of function calls that will cover the blocks listed below them. The results are not perfect and contain a small number of redundancy, but all different scenarios are discovered, including the one where calls of \code{pop()} occur more times than \code{push()}. \\

\begin{listing}
\caption{\label{stack-output}Output of program for the \code{stack} class}
\inputminted{text}{results.txt} \\
\end{listing}
\subsubsection{memory leaks}
\label{sec-2-4-3-2}
Since the program will be compiled using the \code{`-fsanitize=address`} compiler flag to include the library that is responsible for observing coverage. \code{AddressSanitizer} library detects memory issues that your unit might have. For example, if we were to remove the destructor from the \code{stack} implementation, we would get the message shown on Listing \ref{Memory-leak}. \\
\begin{listing}[H]
\begin{minted}[]{text}
==30539==ERROR: LeakSanitizer: detected memory leaks

Direct leak of 10656 byte(s) in 645 object(s) allocated from:
    #0 0x4f2f22 in operator new[](unsigned long) ...
    #1 0x4fe6ae in stack<int>::stack(int) examples/stack.cpp:11:11
    #2 0x4ffcb5 in main::$_1::operator()(int) const ...
    #3 0x4ffc79 in main::$_1::__invoke(int) (bin/main+0x4ffc79)
    #4 0x5015c6 in CombinationTester<stack<int> >::run() ...
    #5 0x4ff6c7 in main (bin/main+0x4ff6c7)
    #6 0x7fa3bbaeeb96 in __libc_start_main ...

SUMMARY: AddressSanitizer: 10656 byte(s) leaked in 645 allocation(s).
\end{minted}
\caption{\label{Memory-leak}Memory leak discovered when the destructor is missing}
\end{listing}

For more information about interpreting and fixing these messages visit documentation for AddressSanitizer\footnote{DEFINITION NOT FOUND.}. \\
\section{Troubleshooting}
\label{sec-2-5}
Following issues might arise during different parts of using the program. \\
\subsection{Problems during installation}
\label{sec-2-5-1}
the tests are designed so that all of the underlying infrastructure will be checked. If you start having any problems: \\
\subsubsection{test if clang build works correctly}
\label{sec-2-5-1-1}
If there were problems with the \code{make integration-test-run} command, it will be because \code{AddressSanitizer} is experiencing issues. \\
\subsection{Problems during runtime}
\label{sec-2-5-2}
\subsubsection{Out of Memory error for AddressSanitizer}
\label{sec-2-5-2-1}
This happens if the source code size is too large, resulting in a lot of guards and callbacks. Please refer to the requirements section. This error could be fixed by advanced tweaking of the \code{AdressSanitizer options}, but as for now is not supported in the project. \\
\chapter{Developer Documentation}
\label{sec-3}
This chapter discusses the structure and contents of the source directory, essential concepts for development and reasoning behind some of the architecture decisions. It also provides overview of the tools necessary for development, along with their usage. \\
\section{Dependencies and internal architecture}
\label{sec-3-1}
Before diving into the specifics, I would like to introduce some material that is necessary for understanding why - and how, everything works. First, I will give a brief overview to the \code{SanitizerCoverage} library and introduce the test class which will illustrate core concepts of both the library and my program. \\
\subsection{SanitizerCoverage library}
\label{sec-3-1-1}
The program relies primarily on LLVM's built in coverage instrumentation to measure coverage of different function cal sequences. Basic understanding of how these functions work is necessary for development. \\

With \code{`-fsanitize-coverage=trace-pc-guard`} flag, the clang compiler will insert the following code on every edge of the control flow. Every edge will have its own guard$_{\text{variable}}$ (uint32$_{\text{t}}$), and in the end the instrumentation will look as given on Listing \ref{guard-example}. Here, "pc" stands for "program counter", and I used this term in the source code and tests as well to describe the parts of the program. \\

\begin{listing}[H]
\begin{minted}[]{c++}
if(*guard)
    __sanitizer_cov_trace_pc_guard(guard);
\end{minted}
\caption{\label{guard-example}How \code{SanitizerCoverage} instrumentation looks}
\end{listing}

There is another function that will be called at least once per dynamic shared object (it may be called more than once with the same values of start/stop). \\

\code{__sanitizer_cov_trace_pc_guard_init(uint32_t *start, uint32_t *stop);} \\

These callbacks are not implemented in the Sanitizer run-time and should be defined by the user. This mechanism is used for fuzzing the Linux kernel, as well as the \code{LibFuzzer} library mentioned earlier. \\

Let me illustrate this using an example. First let us introduce a simple class on Listing \ref{integrationTestClass-definition}. Most functions have descriptive names and their implementation does exactly that, so I will explain only what \code{f4()} does, and also note that \code{f4()} returns a simple integer value without any calculations. \\

\begin{listing}
\caption{\label{integrationTestClass-definition}Definition of the class used in integration tests}
\inputminted[firstline=3,lastline=16]{c++}{test/integrationTestClass.h} \\
\end{listing}

As you will observe on Listing \ref{IntegrationTestClass-implementation}, there are 3 different scenarios for \code{f4()}, each one being a simple \code{return} statement. \code{SanitizerCoverage} will insert guards on entry point of each of them. The rest of the functions will simply have guards inserted in the beginning. \\

\code{SanitizerCoverage} library offers three different levels for observing the control flow of the program. Instrumentation points can be an edge, basic blocks, and function entry points. For this project, using the default edge one was more appropriate, since the target of fuzzing will be small and we can afford detailed coverage. There are different ways offered for tracing data flow, for example callbacks for comparison operations and switches. These could aid in refining the program for more complex applications later. \\

\begin{listing}
\caption{\label{IntegrationTestClass-implementation}Implementation of \code{IntegrationTestClass::f2()} provided in \code{test/integrationTestClass.cc}, with inserted guards highlighted \uline{(TODO)}}
\inputminted[firstline=12,lastline=21]{c++}{test/integrationTestClass.cc} \\
\end{listing}

In the next section, I will continue the \code{IntegrationTestClass} example. It serves as a good demonstration for showing how the program works, before going into the details about each unit. \\
\subsection{Example}
\label{sec-3-1-2}
Let us get back to our test class and think about how we would accomplish to cover all three blocks of \code{f4()}. The first two \code{return} statements are reached if \code{setToggle(true)} has been previously called, since the default value of \code{toggle} is false. Calling \code{f4()} without doing that first results in entering the third branch. \\

\begin{listing}
\caption{\label{integration-test-output}snippet from output of \code{make integration-test-run} shows that the program covered all different cases for \code{IntegrationTestClass}}
\inputminted[lastline=17,fontsize=\small,breaklines]{text}{integration-test-results.txt} \\
\end{listing}

As you see, the program manages to find all of the scenarios that we dicssed, as shown on Listing \ref{integration-test-output}. \\

You will observe that tweaking the number of maximum sequence length will increase the number of covered blocks per unit test (and have larger sequences of function calls) while significantly decreasing the total number of unit tests, until it collapses to very few, maybe even one. It is not reasonable to test the entire logic in a single test, and also not clear when to stop replacing shorter sets of coverage with longer ones (with longer sequences). This also varies depending on the complexity of the unit. However, it is always guaranteed that a longer sequence will never replace a shorter one if their coverage is the same, as observed when tweaking the number from 6 to 7 for \code{IntegrationTestClass}. \\
Right now, it is recommended to start with smaller number of function calls. It is an interesting task to determine the threshold after which having a larger coverage per unit test stops being important. \\

\subsection{Documentation}
\label{sec-3-1-3}
This was a higher-level overview of the core concepts used for the program. The rest of the documentation is provided in the last section of this chapter. All classes are thoroughly documented in header files, and some more detailed explanation and reasoning is sometimes provided in implementation files. For documentation generation, Doxygen library is used with javadoc style. \\

If you have \code{doxygen} installed, Run \code{doxygen Doxyfile} to generate documentation in html and latex source. Latex source needs additional compiling which can be done by running the command \code{make} in the \code{latex} directory. If you would like to change doxygen settings, you can copy the Doxyfile and run \code{doxygen Doxyfile}. \\
\subsection{code conventions}
\label{sec-3-1-4}
Code is formatted according to LLVM coding standard\footnote{DEFINITION NOT FOUND.}. Clang-format\footnote{DEFINITION NOT FOUND.} is used and can be added to commit hook, or alternatively use clang-format plugin for IDE of your choice. \\
\section{Makefile and project layout}
\label{sec-3-2}
As mentioned before, the project is compiled using GNU make, since it was more straightforward to express the different compilation commands and necessary flags for Sanitizer library. Each recipe is self documenting by using @echo to display its purpose when invoked. Since this is the most crucial information, it is highlighted in shell output. After that the compile command is displayed in less bright color, so that the user and developer are not distracted but can still observe which commands are being ran. \\
\subsection{Source directory contents}
\label{sec-3-2-1}
Here is a brief overview of the source directory. Each of the subdirectories is displayed alongside corresponding variable in the Makefile. \\

\subsubsection{\$(INC) - include}
\label{sec-3-2-1-1}
Header files. Definitions for all classes of the project. It also contains definition of the template classes along with their implementation. The \lstinline{lib} subdirectory holds the header file of the testing framework. \\
\subsubsection{\$(SRCDIR) - src}
\label{sec-3-2-1-2}
Implementations of non-template classes from \lstinline{include} directory and a sample for the main file, which should be replaced by user for its own test target unit. \\
\subsubsection{\$(BUILDDIR), \$(TARGETDIR) - build, bin}
\label{sec-3-2-1-3}
Output for binary and object files \\
\subsubsection{\$(TESTDIR) - test}
\label{sec-3-2-1-4}
Test directory. Tests are discussed in detail in \ref{sec-3-3} Testing strategy, but compilation instructios will be provided here. \\
\subsection{Other Makefile variables}
\label{sec-3-2-2}
\subsubsection{\$(SRCEXT)}
\label{sec-3-2-2-1}
Most files in the project end with \code{.cpp} extension. \code{.cc} is reserved for special file types which requre specific compilation instructions. Having different extensions allow to quickly filter out such files. \\
\subsubsection{Object files for classes}
\label{sec-3-2-2-2}
Located in \code{src} directory, ending with \code{.cpp} file extension. Listing \ref{Makefile-main} shows how they are compiled. \\
\subsubsection{\$(CC) and \%(CFLAGS)}
\label{sec-3-2-2-3}
Clang++ is used for all compilation commands, and all possible warnings are turned on and treated as errors in order to ensure good code quality. \\
\subsubsection{\$(INC)}
\label{sec-3-2-2-4}
Adds \code{include} and \code{include/lib} (containing the test library) directories to include path. \\
\subsection{Compilation commands}
\label{sec-3-2-3}
\subsubsection{main and default command}
\label{sec-3-2-3-1}
\code{make run} runs \code{bin/main}, which is compiled by default when running \code{make}, with all necessary parts, shown on Listing \ref{Makefile-main}. These include: \\

\begin{enumerate}
\item Object files from the \code{src} directory
\label{sec-3-2-3-1-1}
\item Program Test target class, compiled with instrumentation
\label{sec-3-2-3-1-2}
Steps discussed in Section \ref{sec-3-1-1} \\
\item Compiler flag for AddressSanitizer
\label{sec-3-2-3-1-3}
This flag is responsible for callbacks using the guards instrumented in the test target object file \\

\begin{listing}
\caption{\label{Makefile-main}compilation commands for \code{bin/main} in \code{Makefile}}
\inputminted[firstline=47,lastline=60,breaklines]{bash}{Makefile} \\
\end{listing}
\end{enumerate}

\subsubsection{docs}
\label{sec-3-2-3-2}
Generates the documentation as described in Section \texttt{documentation} \\
\subsubsection{clean}
\label{sec-3-2-3-3}
Empties the output directories and removes all object / target files. \\
\subsubsection{test and integration test}
\label{sec-3-2-3-4}
These commands are discussed more later in Section \ref{sec-3-3}. \\
\section{Testing strategy}
\label{sec-3-3}
The project is thoroughly tested using the catch2 framework. Tests are represented with Behavior-driven Given-When-Then style, and described scenarios carefully follow documentation. This library was chosen for its minimalistic setup and ability to describe the test cases with full sentences. \\

\subsection{Catch2 library}
\label{sec-3-3-1}
\begin{listing}
\caption{\label{catch-main}contents of \code{test/catch2-main.cpp} defining the main function for test framework}
\inputminted{c++}{test/catch2-main.cpp} \\
\end{listing}

Catch is a multi-paradigm test framework for C++, distributed as a single header file. Though that does not mean that it needs to be compiled into every translation unit.  Since it needs to be defined just once and used for any test case, it is possible and more efficient\footnote{DEFINITION NOT FOUND.} to compile it to an object file which is later included in tests, as displayed on Listings \ref{catch-main} and \ref{catch-compile}. \\

\begin{listing}
\caption{\label{catch-compile}compiling contents of \code{test/catch-main.cpp}}
\inputminted[firstline=70,lastline=75]{bash}{Makefile} \\
\end{listing}

After this, we can compile the test files. Unit tests follow the convention by having the corresponding class name followed by "-test" suffix, and their compilation recipe is shown in Listing \ref{catch-include}. The exception to this is the integration test, which was discussed in Section \ref{sec-3-1-2}. I will provide some more details about it in this section. \\

\begin{listing}
\caption{\label{catch-include}including compiled \code{catch2-main.cpp} in tests and compiling them}
\inputminted[firstline=77,lastline=91,breaklines]{bash}{Makefile} \\
\end{listing}

The library offers two different ways of structuring test cases, and I chose the GIVEN-WHEN-THEN one because of its behavior driven style syntax, which allowed me to express expectations more concisely. \\

\subsection{An example test case}
\label{sec-3-3-2}
The library was created by test driven development, and all features discussed in documentation have a corresponding test case. I will walk through one example in more detail. Unit with most scenarios was \code{CoverageReporter}, since it needs to decide when the new coverage was meaningful/worth storing. I will show how I handled one of the scenarios. \\

Let us discuss what should happen if a new function sequence recorded set of pc blocks that has already been saved earlier. Listing \ref{coveragereporter-test-longer} shows the setup for the scenario and first test when the new coverage is covered by the sequence of same length. \\

\begin{listing}
\caption{\label{coveragereporter-test-longer}Setup for scenario "coverage for new sequence already exists" in \code{test/coverageReporter-test.cpp}, and case when new coverage is reached with longer sequence compared to what was recorded earlier}
\inputminted[firstline=86,lastline=104]{c++}{test/coverageReporter-test.cpp} \\
\end{listing}

\begin{center}
\begin{tabular}{}
\\
\\
\end{tabular}
\end{center}

Same should happen if the sequence is the same size, and that is checked as well. However, if the same coverage was achieved by a shorter sequence, we would like to reflect that. Listing \ref{coveragereporter-test-shorter} shows how I check the behavior. \\

\begin{listing}
\caption{\label{coveragereporter-test-shorter}Case when new coverage has a shorter sequence in "coverage for new sequence already exists" scenario}
\inputminted[firstline=117,lastline=126]{c++}{test/coverageReporter-test.cpp} \\
\end{listing}


I also paid attention to error handling. Listing \ref{test-exception} shows the scenario when the user flushes collected coverage but forgets setting a sequence beforehand. This needs to hold true after flushing any number of times, which is why I first start the coverage and flush correctly, but do not call the \code{startCoverage()} function for the second time. \\

\begin{listing}
\caption{\label{test-exception}Asserting that \code{CoverageReporter} communicates the error in case of developer forgetting to set the sequence.}
\inputminted[firstline=129,lastline=139,breaklines]{c++}{test/coverageReporter-test.cpp} \\
\end{listing}

\subsection{Integration test}
\label{sec-3-3-3}
The example discussed in Section \ref{sec-3-1-2} is used to test the \code{combinationTester} class. Since this class only combines the functionalities of other ones without adding a lot of logic and scenarios, I use the test case to determine that the Sanitizer Coverage library properly works and at least one pc is reported during runtime. The compilation commands are very similar to the ones used for \code{main.cc}, and they are provided in Listing \ref{Makefile-integration-test}. \\

\begin{listing}
\caption{\label{Makefile-integration-test}Compiling the integration test \code{Makefile}}
\inputminted[firstline=97,lastline=118,breaklines]{bash}{Makefile} \\
\end{listing}

\chapter*{References} \\
\setcounter{page}{43} \\
\footnotemark[1]{} Kostya Serebryany, Fuzz or lose! \url{https://github.com/CppCon/CppCon2017/blob/master/Demos/Fuzz\%20Or\%20Lose/Fuzz\%20Or\%20Lose\%20-\%20Kostya\%20Serebryany\%20-\%20CppCon\%202017.pdf} \\

\footnotemark[2]{} Libfuzzer documentation, \url{https://llvm.org/docs/LibFuzzer.html} \\

\footnotemark[3]{} Clang SanitizerCoverage library Documentation, \url{https://clang.llvm.org/docs/SanitizerCoverage.html} \\

\footnotemark[4]{} Techiedelight, Stack Implementation in C++, \url{https://www.techiedelight.com/stack-implementation-in-cpp/} \\

\footnotemark[5]{} Clang AddressSanitizer documentation, \url{https://clang.llvm.org/docs/AddressSanitizer.html} \\

\footnotemark[6]{} LLVM coding standards, \url{https://llvm.org/docs/CodingStandards.html} \\

\footnotemark[8]{} Improving catch library compilation, \url{https://github.com/catchorg/Catch2/blob/master/docs/slow-compiles.md#top} \\

\footnotemark[7]{} Clang-Format documentation, \url{https://clang.llvm.org/docs/ClangFormatStyleOptions.html} \\
% Emacs 25.2.2 (Org mode 8.2.10)
\end{document}
